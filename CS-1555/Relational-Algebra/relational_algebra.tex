\documentclass[12pt, letterpaper]{report}
\usepackage[margin=1in]{geometry}
\usepackage[utf8]{inputenc}
\usepackage{amssymb}
\usepackage{graphicx}
\usepackage{float}
\usepackage{subfig}
\graphicspath{ {./img/} }
\setlength\parindent{0pt}
\renewcommand\thesection{\arabic{section}.}
\renewcommand{\thesubsection}{\alph{subsection})}
\def\ojoin{\setbox0=\hbox{$\bowtie$}
\rule[-.02ex]{.25em}{.4pt}\llap{\rule[\ht0]{.25em}{.4pt}}}
\def\leftouterjoin{\mathbin{\ojoin\mkern-5.8mu\bowtie}}
\def\rightouterjoin{\mathbin{\bowtie\mkern-5.8mu\ojoin}}
\def\fullouterjoin{\mathbin{\ojoin\mkern-5.8mu\bowtie\mkern-5.8mu\ojoin}}


\title{CS1555 - Relational Algebra}
\author{Zachary M. Mattis}


\begin{document}
	
\maketitle


\section{Arity / Cardinality}

% A
\subsection{$\pi_{country}(Users)$}

\begin{verbatim}
Arity       - 1
Cardinality - Min() = 1, Max() = 100
\end{verbatim}

The arity is 1 due to only projecting the \textit{Country} column from the \textit{Users} relation. The min cardinality value occurs when all users live in the same country. The max cardinality occurs when each user lives in a different country (no duplicates). This could have an absolute max value of 195, as there are that many countries in the world today.

% B
\subsection{$Mail \leftouterjoin_{Mail.mailID=Recipients.mailID} Recipients$}

\begin{verbatim}
Arity       - 9
Cardinality - Min() = 3500, Max() = 5499
\end{verbatim}

The arity is 9 due to selecting all columns from both the \textit{Mail} and \textit{Recipients} relations, even if they have the same value (\textit{mailID}). The min cardinality value occurs when each piece of mail has been sent to at least 1 user. The max cardinality occurs if only 1 piece of mail has been sent to 3500 recipients, resulting in 1999 records with null recipients on the left outer join. However, if mail MUST have at least 1 recipient, then the max cardinality is 3500.

% C
\subsection{$Mail * Labels$}

\begin{verbatim}
Arity       - 9
Cardinality - 5,000,000
\end{verbatim}


The arity is 9 due to selecting all columns from both the \textit{Mail} and \textit{Labels} relations, even if they have the same value (\textit{mailID}). The cardinality value is equal to the cardinality of each relation multiplied together.

\section{Expressions}

% A
\subsection{Urgent Mail}

\[ \pi_{fname,\;lname,\;userID,\;subject,\;timeSent} (\sigma_{urgent=1} (Mail\;\leftouterjoin_{Mail.senderID=Users.userID}\;Users) )  \]

\begin{verbatim}
SQL:

SELECT 
    u.fname,
    u.lname,
    u.userID,
    m.subject,
    m.timeSent
FROM
    Mail m
LEFT JOIN Users u
    ON m.senderID = u.userID
WHERE
    m.urgent = 1;

\end{verbatim}


% B
\subsection{Outside USA}

\[ \pi_{userID} (\sigma_{urgent=1\;\wedge\;country <> ``USA"} (Mail\;\leftouterjoin_{Mail.senderID=Users.userID}\;Users) )  \]

\begin{verbatim}
SQL:

SELECT  DISTINCT
    u.userID
FROM
    Mail m
LEFT JOIN Users u
    ON m.senderID = u.userID
WHERE
    m.urgent = 1 AND
    u.country <> "USA";

\end{verbatim}

% C
\subsection{Important Label}

\[ G_{sum(mailID)} (\pi_{mailID} (\sigma_{label=``important"\;\wedge\;urgent<>1\;\wedge\;timeSent=2015} (Mail\;\leftouterjoin_{Mail.mailID=Labels.mailID}\;Labels ))  \]

\begin{verbatim}
SQL:

SELECT 
    COUNT( DISTINCT m.mailID)
FROM
    Mail m
LEFT JOIN Labels l
    ON m.mailID = l.mailID
WHERE
    l.label = "important" AND
    m.urgent <> 1 AND
    YEAR(m.timeSent) = 2015;

\end{verbatim}

% D
\subsection{Ava Lovelace}


\[ \pi_{fname,\;lname} (\sigma_{recipientID=[\pi_{recipientID} (\sigma_{fname=``Ava"\;\wedge\;lname=``Lovelace"} (Mail\;\leftouterjoin\;_{senderID=userID}\;Users\leftouterjoin\;Recipients ))]} \]
\[ (Mail\;\leftouterjoin\;Recipients\;\leftouterjoin\;_{senderID=userID}\;Users ))  \]


\begin{verbatim}
SQL:

SELECT
    u.fname,
    u.lname
FROM
    Mail m
LEFT JOIN Users u
    ON m.senderID = u.userID
LEFT JOIN Recipients r
    ON m.mailID = r.mailID
WHERE
    r.recipientID = (
    
   SELECT DISTINCT
       r.recipientID
   FROM 
       Mail m
   LEFT JOIN Recipients r
       ON m.mailID = r.mailID
   LEFT JOIN Users u
       ON m.senderID = u.userID
   WHERE
       u.fname = "Ava" AND
       u.lname = "Lovelace"

   );

\end{verbatim}

% E
\subsection{Not Reply in USA}


\[ \pi_{mailID} (\sigma_{replyTo=NULL\;\wedge\;(u\_send.country=``USA"\;\vee\;u\_recv.country=``USA")} \]
\[ (Mail\;\leftouterjoin\;Recipients\;\leftouterjoin\;_{senderID=u\_send.userID}\;Users\;u\_send\;\leftouterjoin\;_{recipientID=u\_recv.userID}\;Users\;u\_recv ))  \]

\begin{verbatim}
SQL:

SELECT DISTINCT
    m.mailID
FROM
    Mail m
LEFT JOIN Users u_send
    ON m.senderID = u_send.userID
LEFT JOIN Recipients r
    ON m.mailID = r.mailID
LEFT JOIN user u_recv
    ON r.recipientID = u_recv.userID
WHERE
    m.replyTo IS NULL AND
    ( u_send.country = "USA" OR
      u_recv.country = "USA"
    );

\end{verbatim}

\end{document}